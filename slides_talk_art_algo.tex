\documentclass[12pt,c]{beamer}
\usepackage{graphicx}
\setbeameroption{hide notes}
\setbeamertemplate{note page}[plain]
% get rid of junk
\usetheme{Rochester}
\definecolor{orange-ish}{RGB}{255,86,34}
\setbeamercolor{structure}{fg=orange-ish}
\beamertemplatenavigationsymbolsempty
\hypersetup{pdfpagemode=UseNone} % don't show bookmarks on initial view
% font
\usepackage[utf8]{inputenc}
\usepackage[english]{babel}
\usepackage[T1]{fontenc}
\usepackage[]{algorithm2e}

\setbeamerfont{note page}{family*=pplx,size=\footnotesize} % Palatino for notes
% "TeX Gyre Heros can be used as a replacement for Helvetica"
% In Unix, unzip the following into ~/.fonts
% In Mac, unzip it, double-click the .otf files, and install using "FontBook"
% http://www.gust.org.pl/projects/e-foundry/tex-gyre/heros/qhv2.004otf.zip
% named colors
\definecolor{offwhite}{RGB}{249,242,215}
\definecolor{foreground}{RGB}{255,255,255}
\definecolor{background}{RGB}{30,30,30}
\definecolor{title}{RGB}{0,0,0}
\definecolor{gray}{RGB}{155,155,155}
\definecolor{subtitle}{RGB}{0,0,0}
\definecolor{hilight}{RGB}{102,0,0}
\definecolor{vhilight}{RGB}{255,111,207}
\definecolor{lolight}{RGB}{155,155,155}
%\definecolor{green}{RGB}{125,250,125}
% use those colors
\setbeamercolor{titlelike}{fg=title}
\setbeamercolor{subtitle}{fg=subtitle}
\setbeamercolor{institute}{fg=gray}
\setbeamercolor{normal text}{fg=foreground,bg=background}
\setbeamercolor{item}{fg=foreground} % color of bullets
\setbeamercolor{subitem}{fg=gray}
\setbeamercolor{tableofcontents}{fg = hilight}
\setbeamercolor{itemize/enumerate subbody}{fg=gray}
\setbeamertemplate{itemize subitem}{{\textendash}}
\setbeamerfont{itemize/enumerate subbody}{size=\footnotesize}
\setbeamerfont{itemize/enumerate subitem}{size=\footnotesize}
% page number
\setbeamertemplate{footline}{%
\raisebox{5pt}{\makebox[\paperwidth]{\hfill\makebox[20pt]{\color{gray}
\scriptsize\insertframenumber}}}\hspace*{5pt}}
% add a bit of space at the top of the notes page
\addtobeamertemplate{note page}{\setlength{\parskip}{12pt}}
% a few macros

\newcommand{\bi}{\begin{itemize}}
\newcommand{\ei}{\end{itemize}}
\newcommand{\ig}{\includegraphics}
\renewcommand{\i}{\item[$\cdot$]}
\newcommand{\subt}[1]{{\footnotesize \color{subtitle} {#1}}}
\newcommand{\transti}[1]{\begin{frame} \begin{block}{#1}   \end{block} \end{frame}}
\newcommand{\N}{\mathbb{N}}
% title info
\title{Algorithmic art}
\subtitle{Talktilla}
\date{December 11, 2015}
\author{Alexandre Terrien}
\begin{document}
\maketitle

\section{Introduction to algorithmic art}
\begin{frame}{Credits}
TIPE subject, supervised by Stefan Bornhoffen\\
Participants : 
\bi
\i Vincent Destree
\i Marguerite-Charlotte Peron
\i Maxime Sautereau
\i Alexandre Terrien
\i Alexandre Vincent
\ei
\end{frame}
\begin{frame}{Computer art}
\begin{block}{Definition}
Art where a computer is part of the production or displaying of the artwork.
\end{block}
Examples : 
\bi 
\i Computer-Generated Imagery
\i Video games
\i Interactive Art
\i Generative art
\ei
\end{frame}
\begin{frame}{Algorithmic art}
\begin{block}{Definition}
Art (mainly visual) where the artwork is produced by an algorithm. The artist usually doesn't take part in the generation.
\end{block}
\end{frame}
\begin{frame}{Fractals}
\begin{figure}
\begin{center}
		\ig[scale=0.35]{mandlebrot.jpg}
		\caption{Mandlebrot set, created by Wolfgang Beyer with the program Ultra Fractal 3.}
\end{center}
\end{figure}
\end{frame}
\begin{frame}{L-systems}
\begin{figure}
	\ig[scale=0.18]{lsys.png}
	\caption{L-system modelisation by Sakurambo}
\end{figure}
\end{frame}
\begin{frame}{Our realisation}
Many known algorithms/models to produce such work, so the interesting part is to create an original algorithm.\\[0.5cm]
Representation of waves\\
\bi 
\i Propagation
\i Interactions
\i Colours
\i Rainbows !
\ei
\end{frame}
\section{Technicalities}
\begin{frame}{Data structures : Point}
Attributes : 
\bi 
\i Red channel
\i Green channel
\i Blue channel
\i Alpha channel
\ei
Methods : 
\bi 
\i Associated setters and getters
\i Render a point
\ei
\end{frame}
\begin{frame}{Representation}
Screen represented by matrix of points, each element of the array is linked to a pixel.\\
We iterate over the matrix and compute values for one or more of the channel (in our case, the transparency)
\end{frame}
\begin{frame}{General algorithm}
\begin{algorithm}[H]
\SetAlgoLined
\KwData{buffer, buffer_w, buffer_h}
\While{$\neg\text{ }stopCondition()$}{
$source \leftarrow randomPoint()$\\
	\For{$i\leftarrow 0$ \KwTo $buffer_w$}{
		\For{$i\leftarrow 0$ \KwTo $buffer_h$}{
			$d \leftarrow dist(buffer[i,j], source)$\\
			$buffer[i,j].alpha \leftarrow f(buffer[i,j].alpha, d)$
		}
	}
	$render(buffer)$
}
\end{algorithm}
\end{frame}
\begin{frame}{Choice of function}
\bi
\i Variation of alpha (transparency) value. 
\i Value between 0 and 255
\i Representation of waves $\Rightarrow$ sinus and cosinus
\i Apparition of previous alpha value (interactions)
\i Random parameters for each source
\ei
\end{frame}
\begin{frame}{Results}
\ig[width=1\textwidth]{inter.png}
\end{frame}
\begin{frame}{Results}
\ig[width=\textwidth]{eye.png}
\end{frame}
\begin{frame}{Results}
\ig[width=\textwidth]{bubble.png}
\end{frame}
\begin{frame}{Results}
\ig[width=\textwidth]{raindrop.png}
\end{frame}
\end{document}
